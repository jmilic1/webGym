\chapter{Arhitektura i dizajn sustava}
		
	
	Arhitektura korištena pri izradi WebGym web aplikacije sastoji se od:
	\begin{itemize}
		\item   Web poslužitelj
		\item 	Web aplikacija
		\item 	Baza podataka
		\item   Poslužitelj baze podataka	
	\end{itemize}
	
	\begin{figure}[H]
			\includegraphics[scale=1.0]{slike/arh.PNG} %veličina slike u odnosu na originalnu datoteku i pozicija slike
			\centering
			\caption{Arhitektura sustava}
			\label{fig:promjene}
		\end{figure}

	Na samom početku programiranja sustava bilo je neophodno dogovoriti se oko arhitekture i tehnologija koje će se koristiti za izradu programskog rješenja. Preciznije, dogovaralo se o pojedinostima kao što su programski jezici, razvojna okruženja i koncepti za razvoj aplikacije. Arhitekturu smo odabrali kako bi zadovoljavala mnoge značajke kao što su fleksibilnost, mogućnost nadogradnje i jeftinije održav-\\anje. Web aplikacija je razuman odabir zato što ovisi samo o veličini ekrana uređaja na kojoj se koristi, a taj problem je jednostavniji za riješiti nego problem specifične arhitekture pojedinih operacijskih sustava. Također, odabrana arhitektura od korisnika naše web aplikacije zahtijeva samo pristup internetu te web preglednik.  \newline
	\indent Pregled web-stranice, svih podataka i zatraženih medija odvija se pomoću \linebreak \underbar{\textit{web preglednika}}. Web preglednik ima nekoliko uloga, jedna od njih je prikaz stranice na način koji je čitljiv čovjeku. Stoga, možemo reći da web preglednik ima ulogu, odnosno funkciju prevoditelja. Također, upravo je web preglednik alat pomoću kojeg korisnik šalje zahtjeve web poslužitelju. \newline
	\indent Iduća funkcionalnost je komunkacija korisnika s aplikacijom. Tome služi \linebreak \underbar{\textit{web poslužitelj}} koji je svojevrsni motor web aplikacije. Sama komunikacija odvija se pomoću HTTP protokola koji služi za prijenos informacija i komunikaciju na internetu. \newline
	\indent \underbar{\textit{Web aplikacija}} se pokreće pomoću web poslužitelja kojemu šalje zahtjeve, a web poslužitelj služi za obradu tih zahtjeva. Za prikaz je zaslužan već prije spomenuti web poslužitelj koji vraća korisniku HTML dokument kao odgovor, a čitljivi prikaz korisniku osiguran je pomoću web preglednika. Također, web poslužitelj zahtjeve vezane uz podatke u bazi podataka prosljeđuje prema poslužitelju baze podataka. \newline
	\indent \underbar{\textit{Poslužitelj baze podataka}} ispunjava ulogu komunikacije s bazom podataka. Dak-\\le, on zaprima i obrađuje zahtjeve za podacima poslane iz web aplikacije koje mu prosljeđuje web poslužitelj. Ovisno o vrsti zathjeva, u odgovorima se mogu nalaziti zatraženi podatci ili informacije o uspješnosti nekih operacija, na primjer autentifikacije ili umetanja novih podataka u bazu podataka.\newline
	\indent Aplikacija se sastoji od front-enda i back-enda. S jedne strane, tehnologija koju smo odabrali za izradu back-enda je Java Spring Boot. Konkretno, back-end obuhvaća komunikaciju s bazom podataka (\underbar{\textit{poslužitelj baze podataka}}). S druge strane, za front-end smo odabrali React.js. Front-end obuhvaća prikaz korisničkog sučelja aplikacije (\underbar{\textit{web aplikacija}} i \underbar{\textit{web poslužitelj}}) i slanje zahtjeva za podacima prema back-endu. Arhitektura sustava za koju smo se odlučili je MVC odnosno (Model-View-Controller) koncept. \newline
	\indent Glavna korist MVC kocepta je ta što dijelovi aplikacije nisu usko povezani i omogućavaju se jednostavnije testiranje, a pogreška u jednom od dijelova neće utjecati na rad ostalih. \newline
	MVC koncept sastoji se od:
	\begin{itemize}
		\item   \textbf{Model} - klase koje opisuju same podatke koji se koriste u aplikaciji. Uz to, klase sadrže i logiku vezanu za te podatke. Podaci u modelu ne ovise o korisničkom sučelju. Model prima ulazne podatke od Controllera.

		\item 	\textbf{View} - omogućuje komunikaciju s korisnikom te definira kako će korisničko sučelje biti prikazano korisniku.
		\item 	\textbf{Controller}	- klase koje kontroliraju komunikaciju korisnika sa aplikacijom. To je sloj aplikacije koji obrađuje korisničke zahtjeve. Controller prihvaća sve ulaze i pretvara ih u naredbe za Model i View.   	
	\end{itemize}

	
		\section{Baza podataka}
		Za potrebe WebGym sustava koristit ćemo relacijsku bazu podataka koja nam pojednostavljuje modeliranje odnosa i scenarija koji se u realnosti događaju. Svaka je tablica definirana svojim imenom te skupom atributa koji su nam potrebni za punu funkcionalnost pojma ili odnosa opisanim tablicom. Brza i jednostavna pohrana te pristup samim podacima je zadaća ovog sustava pohranjivanja podataka u bazu. Baza podataka sastoji se od sljedećih entiteta:
		\begin{itemize}
	        	\item 	Korisnik
	        	\item 	Teretana 
	        	\item 	TeretanaLokacija 
	        	\item   Ciljevi
	        	\item 	PlanTreningaIPrehrane
	        	\item 	Članarina
	        	\item 	ČlanarinaKlijent
	        	\item 	Zamolba
	        	\item   PlanKlijent
	        	\item   TeretanaKorisnik
        	\end{itemize}
		
		
		\subsection{Opis tablica}
		\textbf{Korisnik} Ovaj entitet sadržava sve bitne informacije o korisniku WebGym web aplikacije. Sadrži atribute kao što su: korisničko ime, ime, prezime, email, hashiranu lozinku, broj mobitela, broj PayPala, visinu, težinu te ulogu u sustavu. Primarni ključ ovog entiteta je korisničkoIme te je stoga taj atribut boldan u tablici. Uloga definira ovlasti koje korisnik u sustavu i u odnosima ima. Ovaj entitet je u vezi \emph{One-to-Many} s entitetom Ciljevi preko korisničkog imena klijenta koji je i sam korisnik, u vezi s \emph{Many-to-Many} entitetom PlanTreningaIPrehrane preko korisničkog imena trenera koji je i sam korisnik, u vezi \emph{One-to-Many} s entitetom ČlanarinaKlijent preko korisničkog imena klijenta, u vezi \emph{One-to-Many} s entitetom  Zamolba korisničkog imena trenera, u vezi \emph{Many-to-Many} s entitetom PlanKlijent preko korisničkog imena klijenta te u vezi \emph{Many-to-Many} s entitetom TeretanaKorisnik preko korisničkog imena trenera ili voditelja koji su i sami korisnici.
				
				
    			\begin{longtabu} to \textwidth {|X[10, l]|X[6, l]|X[20, l]|}
    					
    				\hline \multicolumn{3}{|c|}{\textbf{Korisnik}}	 \\[3pt] \hline
    				\endfirsthead
    					
    				\hline \multicolumn{3}{|c|}{\textbf{Korisnik}}	 \\[3pt] \hline
    				\endhead
    					
    				\hline 
    				\endlastfoot
    					
    					\textbf{korisničkoIme}  & VARCHAR	&  	jedinstveno ime korisnika 	\\ \hline
    					ime	& VARCHAR & ime korisnika  	\\ \hline 
    					prezime & VARCHAR & prezime korisnika   \\ \hline 
    					email & VARCHAR	&  	email korisnika s kojim je napravio račun	\\ \hline 
    					lozinka	& VARCHAR & hash korisnikove lozinke za prijavu  	\\ \hline
    					brojMobitela	& VARCHAR & korisnikov broj mobitela  	\\ \hline
    					PayPal	& VARCHAR & korisnikov PayPal za plaćanje usluga  	\\ \hline
    					visina	& INT & korisnikova težina  	\\ \hline
    					težina	& INT & korisnikova visina  	\\ \hline
    					uloga	& VARCHAR & korisnikova uloga u sustavu  	\\ \hline
					
					
				\end{longtabu}
				
				\textbf{Teretana} Ovaj entitet sadrži sve bitne informacije o lancu teretani. Sadrži atribute: id, ime, opis te email, a ovaj je entitet u vezi \emph{One-to-Many} s entitetom TeretanaLokacija preko jedinstvenog brojčanog identifikatora teretane te u vezi \emph{One-to-Many} s entitetom članarina preko jedinstvenog brojčanog identifikatora teretane. Primarni ključ ovog entiteta je id te je stoga taj atribut boldan u tablici.
				\begin{longtabu} to \textwidth {|X[10, l]|X[6, l]|X[20, l]|}
    					
    				\hline \multicolumn{3}{|c|}{\textbf{Teretana}}	 \\[3pt] \hline
    				\endfirsthead
    					
    				\hline \multicolumn{3}{|c|}{\textbf{Teretana}}	 \\[3pt] \hline
    				\endhead
    					
    				\hline 
    				\endlastfoot
    					
    					\textbf{id}  & INT	&  	jedinstveni brojčani identifikator teretane 	\\ \hline
    					ime	& VARCHAR & ime teretane  	\\ \hline 
    					opis & VARCHAR & opis teretane   \\ \hline 
    					email & VARCHAR	&  	e mail teretane	\\ \hline 
					
					
				\end{longtabu}
				
			\textbf{TeretanaLokacija} Ovaj entitet sadrži sve bitne informacije o pojedinoj teretani. Sadrži atribute: id, id lanca teretani, država u kojoj se nalazi, grad u kojem se nalazi, ulica u kojoj se nalazi, početak radnog vremena teretane, kraj radnog vremena teretane te broj telefona teretane, a ovaj je entitet u vezi \emph{Many-to-One} s entitetom Teretana preko jedinstvenog brojčanog identifikatora teretane, u vezi \emph{One-to-Many} s entitetom Zamolba preko jedinstvenog brojčanog identifikatora pojedine teretane te je u vezi \emph{One-to-Many} s entitetom TeretanaKorisnik preko jedinstvenog brojčanog identifikatora pojedine teretane. Primarni ključ ovog entiteta je id te je stoga taj atribut boldan u tablici, dok je idTeretana strani ključ te je stoga taj atribut pisan formatom italic u tablici.
			\begin{longtabu} to \textwidth {|X[10, l]|X[6, l]|X[20, l]|}
    					
    				\hline \multicolumn{3}{|c|}{\textbf{TeretanaLokacija}}	 \\[3pt] \hline
    				\endfirsthead
    					
    				\hline \multicolumn{3}{|c|}{\textbf{TeretanaLokacija}}	 \\[3pt] \hline
    				\endhead
    					
    				\hline 
    				\endlastfoot
    					
    					\textbf{id}  & INT	&  	jedinstveni brojčani identifikator pojedine teretane 	\\ \hline
    					\textit{idTeretana}	& INT & jedinstveni brojčani identifikator lanca teretani (Teretana.id)  	\\ \hline 
    					država & VARCHAR & država u kojoj se teretana nalazi   \\ \hline 
    					grad & VARCHAR	&  	grad u kojem se teretana nalazi	\\ \hline 
    					ulica	& VARCHAR & ulica u kojoj se teretana nalazi  	\\ \hline
    					radnoVrijemePočetak	& TIME & vrijeme početka rada teretane  	\\ \hline
    					radnoVrijemeKraj	& TIME & vrijeme kraja rada teretane  	\\ \hline
    					telefon	& VARCHAR & broj na koji se teretana može nazvati  	\\ \hline
					
					
			\end{longtabu}
			
			\textbf{Ciljevi} Ovaj entitet sadrži sve važne informacije o ciljevima samog klijenta. Sadrži atribute: id, korisničko ime klijenta te opis cilja, a ovaj je entitet u vezi \emph{Many-to-One} s entitetom Korisnik preko korisničkog imena.Primarni ključ ovog entiteta je id te je stoga taj atribut boldan u tablici, dok je korisničkoImeKlijent strani ključ te je stoga taj atribut pisan formatom italic u tablici.
			\begin{longtabu} to \textwidth {|X[10, l]|X[6, l]|X[20, l]|}
    					
    				\hline \multicolumn{3}{|c|}{\textbf{Ciljevi}}	 \\[3pt] \hline
    				\endfirsthead
    					
    				\hline \multicolumn{3}{|c|}{\textbf{Ciljevi}}	 \\[3pt] \hline
    				\endhead
    					
    				\hline 
    				\endlastfoot
    					
    					\textbf{id}  & INT	&  	jedinstveni brojčani identifikator cilja 	\\ \hline
    					\textit{korisničkoImeKlijent}	& VARCHAR & jedinstveni identifikator klijenta koji cilj obavlja (Korisnik.korisničkoIme)  	\\ \hline 
    					opis & VARCHAR & opis cilja   \\ \hline 
    					obavljeno & BOOLEAN	& je li cilj obavljen	\\ \hline 
					
					
			\end{longtabu}
			
			
			\textbf{PlanTreningaIPrehrane} Ovaj entitet sadrži sve važne informacije o planu treninga i prehrane. Korisnik može kupiti trening te onda ima individualni pristup trenera, ili može samo kupiti plan te dobiti gotov file sa treninzima i preporučenim načinom prehrane. Sadrži atribute: id, korisničko ime trenera, opis plana, datum početka plana, datum isteka plana te atribut koji nam govori je li trening u pitanju, a ovaj je entitet u vezi \emph{Many-to-Many} s entitetom Korisnik preko korisničkog imena. Primarni ključ ovog entiteta je id te je stoga taj atribut boldan u tablici, dok je korisničkoImeTrener strani ključ te je stoga taj atribut pisan formatom italic u tablici.
			\begin{longtabu} to \textwidth {|X[10, l]|X[6, l]|X[20, l]|}
    					
    				\hline \multicolumn{3}{|c|}{\textbf{PlanTreningaIPrehrane}}	 \\[3pt] \hline
    				\endfirsthead
    					
    				\hline \multicolumn{3}{|c|}{\textbf{PlanTreningaIPrehrane}}	 \\[3pt] \hline
    				\endhead
    					
    				\hline 
    				\endlastfoot
    					
    					\textbf{id}  & INT	&  	jedinstveni brojčani identifikator plana treninga i prehrane 	\\ \hline
    					\textit{korisničkoImeTrener} 	& VARCHAR & jedinstveni identifikator trenera čiji je plan (Korisnik.korisničkoIme)  	\\ \hline 
    					opis & VARCHAR & opis plana treninga i prehrane   \\ \hline 
    					datumPočetka & TIMESTAMP & vrijeme početka plana   \\ \hline
    					datumIsteka & TIMESTAMP & vrijeme kraja plana   \\ \hline
    					cijena & DECIMAL & cijena plana   \\ \hline
    					jeLiTrening & BOOLEAN	& je li plan individalni plan, inače je samo kupljeni gotovi plan treninga i prehrane	\\ \hline
					
					
			\end{longtabu}
			
			\textbf{Članarina} Ovaj entitet sadrži sve važne informacije o članarini u teretani.  Sadrži atribute: id, id pojedine teretane, cijena članarine, opis te trajanje, a ovaj je entitet u vezi \emph{Many-to-One} s entitetom Teretana preko jedinstvenog brojčanog identifikatora teretane te je u vezi \emph{One-to-Many} s entitetom ČlanarinaKlijent preko jedinstvenog brojčanog identifikatora članarine. Primarni ključ ovog entiteta je id te je stoga taj atribut boldan u tablici, dok je idTeretana strani ključ te je stoga taj atribut pisan formatom italic u tablici.
			\begin{longtabu} to \textwidth {|X[10, l]|X[6, l]|X[20, l]|}
    					
    				\hline \multicolumn{3}{|c|}{\textbf{Članarina}}	 \\[3pt] \hline
    				\endfirsthead
    					
    				\hline \multicolumn{3}{|c|}{\textbf{Članarina}}	 \\[3pt] \hline
    				\endhead
    					
    				\hline 
    				\endlastfoot
    					
    					\textbf{id}  & INT	&  	jedinstveni brojčani identifikator članarine 	\\ \hline
    					\textit{idTeretana}  	& INT & jedinstveni brojčani identifikator trenere koja nudi članarinu (Teretana.id)  	\\ \hline
    					cijena & DECIMAL & cijena članarine   \\ \hline
    					opis & VARCHAR & opis članarine (što sve članarina uključuje)   \\ \hline
    					trajanje & INTERVAL & trajanje članarine   \\ \hline
					
					
			\end{longtabu}
			
			\textbf{ČlanarinaKlijent} Ovaj entitet sadrži sve važne informacije o članarini koju klijent posjeduje. Sadrži atribute: id, korisničko ime klijenta, id članarine koju teretana nudi, datum početka te datum isteka trajanja članarine, a ovaj je entitet u vezi \emph{Many-to-One} s entitetom Korisnik preko korisničkog imena klijenta te je u vezi \emph{Many-to-One} s entitetom Članarina preko jedinstvenog brojčanog identifikatora članarine. Primarni ključ ovog entiteta je id te je stoga taj atribut boldan u tablici, dok su korisničkoImeKlijent i idČlanarina strani ključevi te su stoga ti atributi pisani formatom italic u tablici.
			\begin{longtabu} to \textwidth {|X[10, l]|X[6, l]|X[20, l]|}
    					
    				\hline \multicolumn{3}{|c|}{\textbf{ČlanarinaKlijent}}	 \\[3pt] \hline
    				\endfirsthead
    					
    				\hline \multicolumn{3}{|c|}{\textbf{ČlanarinaKlijent}}	 \\[3pt] \hline
    				\endhead
    					
    				\hline 
    				\endlastfoot
    					
    					\textbf{id}  & INT	&  	jedinstveni brojčani identifikator članarine koju klijent posjeduje 	\\ \hline
    					\textit{korisničkoImeKlijent}  	& VARCHAR & korisničko ime klijenta koji posjeduje članarinu (Korisnik.korisničkoIme)  	\\ \hline
    					\textit{idČlanarina}  	& INT & jedinstveni brojčani identifikator članarine (Članarina.id)    \\ \hline
					    datumPočetka & TIMESTAMP & datum početka trajanja članarine   \\ \hline
    					datumIsteka & TIMESTAMP & datum kraja trajanja članarine   \\ \hline
					
			\end{longtabu}
			
			\textbf{Zamolba} Ovaj entitet sadrži sve važne informacije o prijavi za posao koju trener šalje pojedinoj teretani. Sadrži atribute: id, korisničko ime trenera, id pojedine teretane u kojoj se trener za posao prijavljuje, te opis prijave, a ovaj je entitet u vezi \emph{Many-to-One} s entitetom Korisnik preko korisničkog imena trenera te je u vezi \emph{Many-to-One} s entitetom Teretana preko jedinstvenog brojčanog identifikatora teretane. Primarni ključ ovog entiteta je id te je stoga taj atribut boldan u tablici, dok su korisničkoImeTrener i idTeretana strani ključevi te su stoga ti atributi pisani formatom italic u tablici.
			\begin{longtabu} to \textwidth {|X[10, l]|X[6, l]|X[20, l]|}
    					
    				\hline \multicolumn{3}{|c|}{\textbf{Zamolba}}	 \\[3pt] \hline
    				\endfirsthead
    					
    				\hline \multicolumn{3}{|c|}{\textbf{Zamolba}}	 \\[3pt] \hline
    				\endhead
    					
    				\hline 
    				\endlastfoot
    					
    					\textbf{id}  & INT	&  	jedinstveni brojčani identifikator prijave 	\\ \hline
    					\textit{korisničkoImeTrener} 	& VARCHAR & korisničko ime trenera koji se prijavljuje za posao u teretani (Korisnik.korisničkoIme)  	\\ \hline
    					\textit{idTeretana}  	& INT & jedinstveni identifikator teretane u koju se prijavljuje (TeretanaLokacija.id)    \\ \hline
					    opis & VARCHAR & opis prijave za posao   \\ \hline
    					dozvola & BOOLEAN & je li treneru odobren rad u teretani   \\ \hline
					
			\end{longtabu}
			
			\textbf{PlanKlijent} Ovaj entitet sadrži sve važne informacije o planu prehrane i treninga kojeg klijent posjeduje. Sadrži atribute: id, id plana, korisničko ime klijenta, te datum kupnje, a ovaj je entitet u vezi \emph{Many-to-One} s entitetom Korisnik preko korisničkog imena klijenta te je u vezi  \emph{Many-to-One} s entitetom PlanTreningaIPrehrane preko jedinstvenog brojčanog identifikatora plana. Primarni ključ ovog entiteta je id te je stoga taj atribut boldan u tablici, dok su korisničkoImeKlijent i idPlan strani ključevi te su stoga ti atributi pisani formatom italic u tablici.
			\begin{longtabu} to \textwidth {|X[11, l]|X[6, l]|X[20, l]|}
    					
    				\hline \multicolumn{3}{|c|}{\textbf{PlanKlijent}}	 \\[3pt] \hline
    				\endfirsthead
    					
    				\hline \multicolumn{3}{|c|}{\textbf{PlanKlijent}}	 \\[3pt] \hline
    				\endhead
    					
    				\hline 
    				\endlastfoot
    					
    					\textbf{id}  & INT	&  	jedinstveni brojčani identifikator plana kojeg klijent posjeduje 	\\ \hline
    					\textit{idPlan} 	& INT & jedinstveni brojčani identifikator plana treninga i prehrane (PlanTreningaIPrehrane.id)  	\\ \hline
    					\textit{korisničkoImeKlijent}  & VARCHAR & korisničko ime klijenta koji plan posjeduje (Korisnik.korisničkoIme) \\ \hline
					    datumKupnje & TIMESTAMP & datum kupnje plana   \\ \hline
		    \end{longtabu}
			
			\textbf{TeretanaKorisnik} Ovaj entitet sadrži sve važne informacije o korisniku koji radi u teretani kao voditelj ili kao trener. Sadrži atribute: id, id pojedine teretane, korisničko ime korisnika, te datum početka rada, a ovaj je entitet u vezi \emph{Many-to-One} s entitetom Korisnik preko korisničkog imena korisnika te je u vezi \emph{Many-to-One} s entitetom TeretanaLokacija preko jedinstvenog brojčanog identifikatora pojedine teretane. Primarni ključ ovog entiteta je id te je stoga taj atribut boldan u tablici, dok su idTeretana i korisničkoImeKorisnik strani ključevi te su stoga ti atributi pisani formatom italic u tablici.
			\begin{longtabu} to \textwidth {|X[11, l]|X[6, l]|X[20, l]|}
    					
    				\hline \multicolumn{3}{|c|}{\textbf{TeretanaKorisnik}}	 \\[3pt] \hline
    				\endfirsthead
    					
    				\hline \multicolumn{3}{|c|}{\textbf{TeretanaKorisnik}}	 \\[3pt] \hline
    				\endhead
    					
    				\hline 
    				\endlastfoot
    					
    					\textbf{id}  & INT	&  	jedinstveni brojčani identifikator odnosa korisnika i teretane 	\\ \hline
    					\textit{idTeretana} 	& INT & jedinstveni brojčani identifikator plana treninga i prehrane (TeretanaLokacija.id)  	\\ \hline
    					\textit{korisničkoImeKorisnik}  & VARCHAR & korisničko ime trenera ili voditelja u teretani (Korisnik.korisničkoIme)   \\ \hline
					    datumPočetkaRada & TIMESTAMP & datum zaposlenja trenera u teretani   \\ \hline
					
			\end{longtabu}
			
			
			\subsection{Dijagram baze podataka}
				
				\begin{figure}[H]
					\hspace*{-1.5cm}
					\includegraphics[scale=0.4]{dijagrami/dijagram_baze_podataka.PNG} %veličina slike u odnosu na originalnu datoteku i pozicija slike
					\centering
					\caption{E-R dijagram baze podataka}
					\label{fig:promjene}
				\end{figure}
			
			\eject
			
			
		\section{Dijagram razreda}
		\begin{figure}[!htb]
		    Na slikama 4.3, 4.4 i 4.5 su prikazani razredi koji pripadaju Controller i Model dijelovima MVC
            arhitekture. Razredi prikazani na slici 4.3 prikazuju razrede koji nasljeđuju Controller. Unutar implementiranih controllera koriste se metode pomoću kojih se manipulira i "razgovara" s DTO (data transfer object). Funkcionalnost metoda unutar pojedinih controllera isprogramirana je da vraća JSON datoteke s html status kodom. Razredi su podijeljeni logički po imenima aktora u našoj web aplikaciji te implementirani s pripadajućim metodama , ovisno o ovlasti pojedinačnog aktora.
            Razredi prikazani na slici 4.4 predstavljaju DTO (data transfer object) razred. U području programiranja DTO je objekt koji prenosi podatke između procesa. Motivacija za njegovu upotrebu je ta da se komunikacija između procesa obično vrši pribjegavanjem udaljenim sučeljima (npr. Web uslugama), gdje je svaki poziv skupa operacija. Rješenje za smanjenje poziva je enkapsulacija  podataka u vrijednosni objekt koji se može prenositi preko mreže, odnosno korištenje DTO-a.
		\end{figure}
		
		\begin{figure}
			\includegraphics[height= 13cm,width=1.2\textwidth]{dijagrami/ControllerClass.jpg}
			\begin{center}
			    Slika 4.3: Dijagram razreda - dio Controllers
			\end{center}
		\end{figure}
		\begin{figure}
			\includegraphics[height= 13cm,width=1.2\textwidth]{dijagrami/DTOClass.png}
			\begin{center}
			    Slika 4.4: Dijagram razreda - dio Data transfer objects
			\end{center}
		\end{figure}
		\begin{figure}
		    Model razredi preslikavaju strukturu baze podataka u aplikaciji. Implementirane metode direktno komuniciraju s bazom podataka te vraćaju tražene podatke. Razred Korisnik predstavlja uniju svih različitih entiteta koji se mogu registrirati u aplikaciji. Razred klijent predstavlja registriranog korisnika te on može koristiti sve rezervirane funkcionalnosti. Razred Administrator predstavlja administratora koji može koristiti više manje sve metode i ima najveće ovlasti te u suštini on upravlja cijelom aplikacijom. Razred Trener predstavlja trenera koji može raditi u nekoj od ponuđenih teretana te koristiti teretanu kao platformu za prodavanje svojih planova treninga i prehrane. Razred Voditelj predstavlja voditelja teretane, voditelj koristi svoje ovlasti kako bi osigurao pravilnu povezanost teretane , trenera i klijenata. Razred Zamolba predstavlja trenerov upit odnosno zamolbu za rad u određenoj teretani. Razred Ciljevi predstavlja određeni cilj pojedinog klijenta. Razred PlanTreningaIPrehrane predstavlja trenerov proizvod koji se nalazi u teretani u kojoj trener radi te ga mogu kupiti trenerovi klijenti.
		\end{figure}
		\begin{figure}
			\includegraphics[height= 20cm,width=1.2\textwidth]{dijagrami/modelsClass.jpg}
			\textbf{Slika 4.5: Dijagram razreda - dio Models}
		\end{figure}
	
		\eject
	
			\eject
		
		\section{Dijagram stanja}
			
			
			\textbf{\textit{dio 2. revizije}}\\
			
			\textit{Potrebno je priložiti dijagram stanja i opisati ga. Dovoljan je jedan dijagram stanja koji prikazuje \textbf{značajan dio funkcionalnosti} sustava. Na primjer, stanja korisničkog sučelja i tijek korištenja neke ključne funkcionalnosti jesu značajan dio sustava, a registracija i prijava nisu. }
			
			
			\eject 
		
		\section{Dijagram aktivnosti}
			
			\textbf{\textit{dio 2. revizije}}\\
			
			 \textit{Potrebno je priložiti dijagram aktivnosti s pripadajućim opisom. Dijagram aktivnosti treba prikazivati značajan dio sustava.}
			
			\eject
		\section{Dijagram komponenti}
		
			\textbf{\textit{dio 2. revizije}}\\
		
			 \textit{Potrebno je priložiti dijagram komponenti s pripadajućim opisom. Dijagram komponenti treba prikazivati strukturu cijele aplikacije.}