\chapter{Arhitektura i dizajn sustava}
		
		\textbf{\textit{dio 1. revizije}}\\

		\textit{ Potrebno je opisati stil arhitekture te identificirati: podsustave, preslikavanje na radnu platformu, spremišta podataka, mrežne protokole, globalni upravljački tok i sklopovsko-programske zahtjeve. Po točkama razraditi i popratiti odgovarajućim skicama:}
	\begin{itemize}
		\item 	\textit{izbor arhitekture temeljem principa oblikovanja pokazanih na predavanjima (objasniti zašto ste baš odabrali takvu arhitekturu)}
		\item 	\textit{organizaciju sustava s najviše razine apstrakcije (npr. klijent-poslužitelj, baza podataka, datotečni sustav, grafičko sučelje)}
		\item 	\textit{organizaciju aplikacije (npr. slojevi frontend i backend, MVC arhitektura) }		
	\end{itemize}

	
		\section{Baza podataka}
		Za potrebe WebGym sustava koristit ćemo relacijsku bazu podataka koja nam pojednostavljuje modeliranje odnosa i scenarija koji se u realnosti dešavaju. Svaka je tablica definirana svojim imenom te skupom atributa koji su nam potrebni za punu funkcionalnost pojma ili odnosa opisanim tablicom. Brza i jednostavna pohrana te pristup samim podacima je zadaća ovog sustava pohranjivanja podataka u bazu. Baza podataka sastoji se od sljedećih entiteta:
		\begin{itemize}
	        	\item 	Korisnik
	        	\item 	Teretana 
	        	\item 	TeretanaLokacija 
	        	\item   Ciljevi
	        	\item 	PlanTreningaIPrehrane
	        	\item 	Članarina
	        	\item 	ČlanarinaKlijent
	        	\item 	Zamolba
	        	\item   PlanKlijent
	        	\item   TeretanaKorisnik
        	\end{itemize}
		
		
		\subsection{Opis tablica}
		\textbf{Korisnik} Ovaj entitet sadržava sve bitne informacije o korisniku WebGym web aplikacije. Sadrži atribute kao što su: korisničko ime, ime, prezime, email, hashiranu lozinku, broj mobitela, broj PayPala, visinu, težinu te ulogu u sustavu. U ulozi ovise ovlasti koje korisnik u sustavu i u odnosima ima. Ovaj entitet je u vezi \emph{One-to-Many} s entitetom Ciljevi preko korisničkog imena klijenta koji je i sam korisnik, u vezi s \emph{Many-to-Many} entitetom PlanTreningaIPrehrane preko korisničkog imena trenera koji je i sam korisnik, u vezi \emph{One-to-Many} s entitetom ČlanarinaKlijent preko korisničkog imena klijenta, u vezi \emph{One-to-Many} s entitetom  Zamolba korisničkog imena trenera, u vezi \emph{Many-to-Many} s entitetom PlanKlijent preko korisničkog imena klijenta te u vezi \emph{Many-to-Many} s entitetom TeretanaKorisnik preko korisničkog imena trenera ili voditelja koji su i sami korisnici.
				
				
    			\begin{longtabu} to \textwidth {|X[6, l]|X[6, l]|X[20, l]|}
    					
    				\hline \multicolumn{3}{|c|}{\textbf{Korisnik}}	 \\[3pt] \hline
    				\endfirsthead
    					
    				\hline \multicolumn{3}{|c|}{\textbf{Korisnik}}	 \\[3pt] \hline
    				\endhead
    					
    				\hline 
    				\endlastfoot
    					
    					\cellcolor{LightGreen}korisničkoIme  & VARCHAR	&  	jedinstveno ime korisnika 	\\ \hline
    					ime	& VARCHAR & ime korisnika  	\\ \hline 
    					prezime & VARCHAR & prezime korisnika   \\ \hline 
    					email & VARCHAR	&  	e mail korisnika s kojim je napravio račun	\\ \hline 
    					lozinka	& VARCHAR & hash korisnikove lozinke za prijavu  	\\ \hline
    					brojMobitela	& VARCHAR & korisnikov broj mobitela  	\\ \hline
    					PayPal	& VARCHAR & korisnikov PayPal za plaćanje usluga  	\\ \hline
    					visina	& INT & korisnikova težina  	\\ \hline
    					težina	& INT & korisnikova visina  	\\ \hline
    					uloga	& VARCHAR & korisnikova uloga u sustavu  	\\ \hline
					
					
				\end{longtabu}
				
				\textbf{Teretana} ovaj entitet sadrži sve bitne informacije o lancu teretani. Sadrži atribute: id, ime, opis te email, a ovaj je entitet u vezi \emph{One-to-Many} s entitetom TeretanaLokacija preko jedinstvenog brojčanog identifikatora teretane te u vezi \emph{One-to-Many} s entitetom članarina preko jedinstvenog brojčanog identifikatora teretane.
				\begin{longtabu} to \textwidth {|X[6, l]|X[6, l]|X[20, l]|}
    					
    				\hline \multicolumn{3}{|c|}{\textbf{Teretana}}	 \\[3pt] \hline
    				\endfirsthead
    					
    				\hline \multicolumn{3}{|c|}{\textbf{Teretana}}	 \\[3pt] \hline
    				\endhead
    					
    				\hline 
    				\endlastfoot
    					
    					\cellcolor{LightGreen}id  & INT	&  	jedinstveni brojčani identifikator teretane 	\\ \hline
    					ime	& VARCHAR & ime teretane  	\\ \hline 
    					opis & VARCHAR & opis teretane   \\ \hline 
    					email & VARCHAR	&  	e mail teretane	\\ \hline 
					
					
				\end{longtabu}
				
			\textbf{TeretanaLokacija} ovaj entitet sadrži sve bitne informacije o pojedinoj teretani. Sadrži atribute: id,id lanca teretani, država u kojoj se nalazi, grad u kojem se nalazi, ulica u kojoj se nalazi, početak ranog vremena teretane, kraj radnog vremena teretane te broj telefona teretane, a ovaj je entitet u vezi \emph{Many-to-One} s entitetom Teretana preko jedinstvenog brojčanog identifikatora teretane, u vezi \emph{One-to-Many} s entitetom Zamolba preko jedinstvenog brojčanog identifikatora pojedine teretane te je u vezi \emph{One-to-Many} s entitetom TeretanaKorisnik preko jedinstvenog brojčanog identifikatora pojedine teretane.
			\begin{longtabu} to \textwidth {|X[6, l]|X[6, l]|X[20, l]|}
    					
    				\hline \multicolumn{3}{|c|}{\textbf{TeretanaLokacija}}	 \\[3pt] \hline
    				\endfirsthead
    					
    				\hline \multicolumn{3}{|c|}{\textbf{TeretanaLokacija}}	 \\[3pt] \hline
    				\endhead
    					
    				\hline 
    				\endlastfoot
    					
    					\cellcolor{LightGreen}id  & INT	&  	jedinstveni brojčani identifikator pojedine teretane 	\\ \hline
    					\cellcolor{LightBlue} idTeretana	& INT & jedinstveni brojčani identifikator lanca teretani (Teretana.id)  	\\ \hline 
    					država & VARCHAR & država u kojoj se teretana nalazi   \\ \hline 
    					grad & VARCHAR	&  	grad u kojem se teretana nalazi	\\ \hline 
    					ulica	& VARCHAR & ulica u kojoj se teretana nalazi  	\\ \hline
    					radnoVrijemePoc	& TIME & vrijeme početka rada teretane  	\\ \hline
    					radnoVrijemeKraj	& TIME & vrijeme kraja rada teretane  	\\ \hline
    					telefon	& VARCHAR & broj na koji se teretana može nazvati  	\\ \hline
					
					
			\end{longtabu}
			
			\textbf{Ciljevi} ovaj entitet sadrži sve važne informacije o ciljevima samog klijenta. Sadrži atribute: id,korisničko ime klijenta te opis cilja, a ovaj je entitet u vezi \emph{Many-to-One} s entitetom Korisnik preko korisničkog imena.
			\begin{longtabu} to \textwidth {|X[6, l]|X[6, l]|X[20, l]|}
    					
    				\hline \multicolumn{3}{|c|}{\textbf{Ciljevi}}	 \\[3pt] \hline
    				\endfirsthead
    					
    				\hline \multicolumn{3}{|c|}{\textbf{Ciljevi}}	 \\[3pt] \hline
    				\endhead
    					
    				\hline 
    				\endlastfoot
    					
    					\cellcolor{LightGreen}id  & INT	&  	jedinstveni brojčani identifikator cilja 	\\ \hline
    					\cellcolor{LightBlue} korisničkoImeKlijent	& VARCHAR & jedinstveni brojčani identifikator klijenta koji cilj obavlja (Korisnik.korisničkoIme)  	\\ \hline 
    					opis & VARCHAR & opis cilja   \\ \hline 
    					obavljeno & BOOLEAN	& je li cilj obavljen	\\ \hline 
					
					
			\end{longtabu}
			
			
			\textbf{PlanTreningaIPrehrane} ovaj entitet sadrži sve važne informacije o planu treninga i prehrane. Korisnik može kupiti trening te onda ima individualni pristup trenera, ili može samo kupiti plan te dobiti gotov file sa treninzima i preporučenim načinom prehrane. Sadrži atribute: id,korisničko ime trenera, opis plana, datum početka plana, datum isteka plana te atribut koji nam govori je li trening u pitanju, a ovaj je entitet u vezi \emph{Many-to-Many} s entitetom Korisnik preko korisničkog imena.
			\begin{longtabu} to \textwidth {|X[6, l]|X[6, l]|X[20, l]|}
    					
    				\hline \multicolumn{3}{|c|}{\textbf{PlanTreningaIPrehrane}}	 \\[3pt] \hline
    				\endfirsthead
    					
    				\hline \multicolumn{3}{|c|}{\textbf{PlanTreningaIPrehrane}}	 \\[3pt] \hline
    				\endhead
    					
    				\hline 
    				\endlastfoot
    					
    					\cellcolor{LightGreen}id  & INT	&  	jedinstveni brojčani identifikator plana treninga i prehrane 	\\ \hline
    					\cellcolor{LightBlue} korisničkoImeTrener 	& VARCHAR & jedinstveni brojčani identifikator trenera čiji je plan (Korisnik.korisničkoIme)  	\\ \hline 
    					opis & VARCHAR & opis plana treninga i prehrane   \\ \hline 
    					datumPočetka & TIMESTAMP & vrijeme početka plana   \\ \hline
    					datumIsteka & TIMESTAMP & vrijeme kraja plana   \\ \hline
    					cijena & DECIMAL & cijena plana   \\ \hline
    					jeLiTrening & BOOLEAN	& je li plan individalni plan, inače je samo kupljeni gotovi plan treninga i prehrane	\\ \hline
					
					
			\end{longtabu}
			
			\textbf{Članarina} ovaj entitet sadrži sve važne informacije o članarini u teretani.  Sadrži atribute: id,id pojedine teretane, cijena članarine, opis te trajanje, a ovaj je entitet u vezi \emph{Many-to-One} s entitetom Teretana preko jedinstvenog brojčanog identifikatora teretane te je u vezi \emph{One-to-Many} s entitetom ČlanarinaKlijent preko jedinstvenog brojčanog identifikatora članarine.
			\begin{longtabu} to \textwidth {|X[6, l]|X[6, l]|X[20, l]|}
    					
    				\hline \multicolumn{3}{|c|}{\textbf{Članarina}}	 \\[3pt] \hline
    				\endfirsthead
    					
    				\hline \multicolumn{3}{|c|}{\textbf{Članarina}}	 \\[3pt] \hline
    				\endhead
    					
    				\hline 
    				\endlastfoot
    					
    					\cellcolor{LightGreen}id  & INT	&  	jedinstveni brojčani identifikator članarine 	\\ \hline
    					\cellcolor{LightBlue} idTeretana  	& INT & jedinstveni brojčani identifikator trenere koja nudi članarinu (Teretana.id)  	\\ \hline
    					cijena & DECIMAL & cijena članarine   \\ \hline
    					opis & VARCHAR & opis članarine (što se članarina uključuje)   \\ \hline
    					trajanje & INTERVAL & trajanje članarine   \\ \hline
					
					
			\end{longtabu}
			
			\textbf{ČlanarinaKlijent} ovaj entitet sadrži sve važne informacije o članarini koju klijent posjeduje. Sadrži atribute: id,korisničko ime klijenta, id članarine koju teretana nudi, datum početka te dazum isteka trajanja članarine, a ovaj je entitet u vezi \emph{Many-to-One} s entitetom Korisnik preko korisničkog imena klijenta te je u vezi \emph{Many-to-One} s entitetom Članarina preko jedinstvenog brojčanog identifikatora članarine.
			\begin{longtabu} to \textwidth {|X[6, l]|X[6, l]|X[20, l]|}
    					
    				\hline \multicolumn{3}{|c|}{\textbf{ČlanarinaKlijent}}	 \\[3pt] \hline
    				\endfirsthead
    					
    				\hline \multicolumn{3}{|c|}{\textbf{ČlanarinaKlijent}}	 \\[3pt] \hline
    				\endhead
    					
    				\hline 
    				\endlastfoot
    					
    					\cellcolor{LightGreen}id  & INT	&  	jedinstveni brojčani identifikator članarine koju klijent posjeduje 	\\ \hline
    					\cellcolor{LightBlue} korisničkoImeKlijent  	& VARCHAR & korisničko ime klijenta koji posjeduje članarinu (Korisnik.korisničkoIme)  	\\ \hline
    					\cellcolor{LightBlue} idČlanarina  	& INT & jedinstveni brojčani identifikator članarine (Članarina.id)    \\ \hline
					    datumPočetka & TIMESTAMP & datum početka trajanja članarine   \\ \hline
    					datumIsteka & TIMESTAMP & datum kraja trajanja članarine   \\ \hline
					
			\end{longtabu}
			
			\textbf{Zamolba} ovaj entitet sadrži sve važne informacije o prijavi za posao koju trener šalje pojedinoj teretani. Sadrži atribute: id,korisničko ime trenera, id pojedine teretane u kojoj se trener za posao prijavljuje, te opis prijave, a ovaj je entitet u vezi \emph{Many-to-One} s entitetom Korisnik preko korisničkog imena trenera te je u vezi \emph{Many-to-One} s entitetom Teretana preko jedinstvenog brojčanog identifikatora teretane.
			\begin{longtabu} to \textwidth {|X[6, l]|X[6, l]|X[20, l]|}
    					
    				\hline \multicolumn{3}{|c|}{\textbf{Zamolba}}	 \\[3pt] \hline
    				\endfirsthead
    					
    				\hline \multicolumn{3}{|c|}{\textbf{Zamolba}}	 \\[3pt] \hline
    				\endhead
    					
    				\hline 
    				\endlastfoot
    					
    					\cellcolor{LightGreen}id  & INT	&  	jedinstveni brojčani identifikator prijave 	\\ \hline
    					\cellcolor{LightBlue} korisničkoImeTrener 	& VARCHAR & korisničko ime trenera koji se prijavljuje za posao u teretani (Korisnik.korisničkoIme)  	\\ \hline
    					\cellcolor{LightBlue} idTeretana  	& INT & jedinstveni identifikator teretane u koju se prijavljuje (TeretanaLokacija.id)    \\ \hline
					    opis & VARCHAR & opis prijave za posao   \\ \hline
    					dozvola & BOOLEAN & je li treneru odobren rad u teretani   \\ \hline
					
			\end{longtabu}
			
			\textbf{PlanKlijent} ovaj entitet sadrži sve važne informacije o planu prehrane i treninga kojeg klijent posjeduje. Sadrži atribute: id, id plana, korisničko ime klijenta, te datum kupnje, a ovaj je entitet u vezi \emph{Many-to-One} s entitetom Korisnik preko korisničkog imena klijenta te je u vezi  \emph{Many-to-One} s entitetom PlanTreningaIPrehrane preko jedinstvenog brojčanog identifikatora plana.
			\begin{longtabu} to \textwidth {|X[6, l]|X[6, l]|X[20, l]|}
    					
    				\hline \multicolumn{3}{|c|}{\textbf{PlanKlijent}}	 \\[3pt] \hline
    				\endfirsthead
    					
    				\hline \multicolumn{3}{|c|}{\textbf{PlanKlijent}}	 \\[3pt] \hline
    				\endhead
    					
    				\hline 
    				\endlastfoot
    					
    					\cellcolor{LightGreen}id  & INT	&  	jedinstveni brojčani identifikator plana kojeg klijent posjeduje 	\\ \hline
    					\cellcolor{LightBlue} idPlan 	& INT & jedinstveni brojčani identifikator plana treninga i prehrane (PlanTreningaIPrehrane.id)  	\\ \hline
    					\cellcolor{LightBlue} korisničkoImeKlijent  & VARCHAR & korisničko ime klijenta koji plan posjeduje (Korisnik.korisničkoIme) \\ \hline
					    datumKupnje & TIMESTAMP & datum kupnje plana   \\ \hline
		    \end{longtabu}
			
			\textbf{TeretanaKorisnik} ovaj entitet sadrži sve važne informacije o korisniku koji radi u teretani kao voditelj ili kao trener. Sadrži atribute: id, id pojedine teretane, korisničko ime korisnika, te datum početka rada, a ovaj je entitet u vezi \emph{Many-to-One} s entitetom Korisnik preko korisničkog imena korisnika te je u vezi \emph{Many-to-One} s entitetom TeretanaLokacija preko jedinstvenog brojčanog identifikatora pojedine teretane.
			\begin{longtabu} to \textwidth {|X[6, l]|X[6, l]|X[20, l]|}
    					
    				\hline \multicolumn{3}{|c|}{\textbf{TeretanaKorisnik}}	 \\[3pt] \hline
    				\endfirsthead
    					
    				\hline \multicolumn{3}{|c|}{\textbf{TeretanaKorisnik}}	 \\[3pt] \hline
    				\endhead
    					
    				\hline 
    				\endlastfoot
    					
    					\cellcolor{LightGreen}id  & INT	&  	jedinstveni brojčani identifikator odnosa korisnika i teretane 	\\ \hline
    					\cellcolor{LightBlue} idTeretana 	& INT & jedinstveni brojčani identifikator plana treninga i prehrane (TeretanaLokacija.id)  	\\ \hline
    					\cellcolor{LightBlue} korisničkoImeKorisnik  & VARCHAR & korisničko ime trenera ili voditelja u teretani (Korisnik.korisničkoIme)   \\ \hline
					    datumPočetkaRada & TIMESTAMP & datum zaposlenja trenera u teretani   \\ \hline
					
			\end{longtabu}
			
			
			\subsection{Dijagram baze podataka}
				\textit{ U ovom potpoglavlju potrebno je umetnuti dijagram baze podataka. Primarni i strani ključevi moraju biti označeni, a tablice povezane. Bazu podataka je potrebno normalizirati. Podsjetite se kolegija "Baze podataka".}
			
			\eject
			
			
		\section{Dijagram razreda}
		
			\textit{Potrebno je priložiti dijagram razreda s pripadajućim opisom. Zbog preglednosti je moguće dijagram razlomiti na više njih, ali moraju biti grupirani prema sličnim razinama apstrakcije i srodnim funkcionalnostima.}\\
			
			\textbf{\textit{dio 1. revizije}}\\
			
			\textit{Prilikom prve predaje projekta, potrebno je priložiti potpuno razrađen dijagram razreda vezan uz \textbf{generičku funkcionalnost} sustava. Ostale funkcionalnosti trebaju biti idejno razrađene u dijagramu sa sljedećim komponentama: nazivi razreda, nazivi metoda i vrste pristupa metodama (npr. javni, zaštićeni), nazivi atributa razreda, veze i odnosi između razreda.}\\
			
			\textbf{\textit{dio 2. revizije}}\\			
			
			\textit{Prilikom druge predaje projekta dijagram razreda i opisi moraju odgovarati stvarnom stanju implementacije}
			
			
			
			\eject
		
		\section{Dijagram stanja}
			
			
			\textbf{\textit{dio 2. revizije}}\\
			
			\textit{Potrebno je priložiti dijagram stanja i opisati ga. Dovoljan je jedan dijagram stanja koji prikazuje \textbf{značajan dio funkcionalnosti} sustava. Na primjer, stanja korisničkog sučelja i tijek korištenja neke ključne funkcionalnosti jesu značajan dio sustava, a registracija i prijava nisu. }
			
			
			\eject 
		
		\section{Dijagram aktivnosti}
			
			\textbf{\textit{dio 2. revizije}}\\
			
			 \textit{Potrebno je priložiti dijagram aktivnosti s pripadajućim opisom. Dijagram aktivnosti treba prikazivati značajan dio sustava.}
			
			\eject
		\section{Dijagram komponenti}
		
			\textbf{\textit{dio 2. revizije}}\\
		
			 \textit{Potrebno je priložiti dijagram komponenti s pripadajućim opisom. Dijagram komponenti treba prikazivati strukturu cijele aplikacije.}