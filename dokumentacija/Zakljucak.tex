\chapter{Zaključak i budući rad}
		
		\textbf{}\\
		  Zadatak naše grupe bio je napraviti web aplikaciju za  pronalazak teretana i njihovih informacija u svrhu učlanjivanja ili pronalaženja trenera i trenerovih ponuda za programe treniranja i planova prehrana. Mogućnosti ove aplikacije su velike te mogu značajno olakšati odabir teretane te vođenje većine bitnih stvari povezanih s teretanom. Na izradi aplikacije sudjelovalo je 7 osoba u trajanju od 17 tjedana. Tim je bio podijeljen u 3 tima koja su djelovali kao homogena zajednica i uz kontinuiranu međusobnu komunukaciju stvorili završni proizvod. Rad na projektu možemo opisati kroz dvije veće faze. 
		  
		  U prvoj fazi projekta tim se dogovarao o tehnologijama i tehničkoj podršci koju ćemo koristiti za izradu ove aplikacije, uz dogovore potrošen je dio vremena za idividualno učenje i usavršavanje spomenutih tehnologija. Nakon detaljnih upoznavanja, kako sa tehnologijama tako i među timom, krenuli smo u stvaranje aplikacije. Glavni cilj prve faze bio nam je stvoriti kostur projekta odnosno napraviti dobru dokumentaciju po kojoj ćemo onda graditi našu aplikaciju. 
		  
		  U drguoj fazi projekta dolazi do izražaja pravilan raspored timova na određenim područjima, tako smo imali podtim za backend, za frontend i za dovršavanje dokumentacije. Unatoč velikoj organiziranosti i pravilnoj podjeli posla u drugoj fazi projekta nailazili smo na dosta problema vezanih uz tehnologiju, te je u tim trenutcima do izražaja došao samostalni rad na učenju dodatnih materijala i alata kako bi riješili ove probleme. Najčešći problemi na koje smo nailazili bili su vezani za radni okvir React s kojim se većina tima susrela tek na ovom projektu. Tijekom izrade projekta shvatili smo da je ključan dio izrade tehničke aplikacije bio pravilna i transparentna dokumentacija te kontinuirana komunikacija između podtimova. Komunikaciju smo vršili preko MS Teamsa koji je omogućio lagane i brze sastanke za probleme koje je trebalo riješiti. 
		  
		  Sudjelovanje i rad na ovom projektu stvorilo je nove poglede na rad u timu te je stečeno vrijedno radno iskustvo u stvaranju web aplikacije. Za kraj možemo reći da su pozitivne strane projekta to što smo se naučili služiti nekim tehnologijama s kojima se do sada nismo susretali te smo dobili osjećaj za to koliko je dobra i precizna dokumentacija bitna za uspješnost projekta, čak i na ovako relativno jednostavnom projektu. 
		  
		  Najviše problema imali smo s nekonzistentnom upotrebom nekih domenskih pojmova, tako su se do sredine prve faze projekta miješali pojmovi teretane i lokacije određene teretane te su različiti članovi tima imali različite interpretacije tih pojmova. Također smo pri definiranju putanja između fronenda i backenda trebali biti nešto precizniji jer nam je nedovoljno kvalitetna definicija te komunikacije na trenutke stvarala probleme.
		\eject 