\chapter{Opis projektnog zadatka}
		
				
		Cilj ovog projekta je razviti programsku podršku za stvaranje web aplikacije
		\textit{"WebGym"} koja će svojim korisnicima uvelike olakšati administrativne 
		poslove vezane uz njihove odlaske u teretanu. Cilj platforme \textit{"WebGym"} 
		je poboljšavanje ukupnog doživljaja teretane, a namijenjena je neregistriranim 
		korisnicima, registriranim korisnicima, trenerima u teretani i voditeljima svake od 
		teretana (jedan voditelj može biti zadužen za više
		teretana i za svaku teretanu može biti zaduženo više voditelja). Za registraciju bilo kojeg od korisnika potrebno je unijeti
		ime, prezime, email adresu i osobne podatke bitne za trenere (visina, težina…) te je PayPal račun opcionalan. 
		
		Prilikom pokretanja sustava (odlaska na početnu stranicu) ukoliko korisnik nije trenutno prijavljen pokazuje se općenita početna stranica. Na toj početnoj stranici u gornjem desnom kutu nalaze se polja za unos korisničkog imena i lozinke te gumb za registraciju korisnika koji se do sada nisu registrirali. Na početnoj stranici se za neregistrirane korisnike nalazi popis najpopularnijih teretana, a svaka teretana u tom popisu ima prikazanu svoju profilnu sliku, ime i adresu teretane.
		
		Za kreiranje novog računa potrebni su sljedeći podaci:

		\begin{packed_item}
			\item Korisničko ime
			\item Ime
			\item Prezime
			\item Broj mobitela
			\item e-mail
		\end{packed_item}
	
		Također opcionalno se mogu unijeti i sljedeći podatci:
		
		\begin{packed_item}
			\item PayPal račun
			\item Visina
			\item Težina
		\end{packed_item}
	
		Pri izradi korisničkog računa moguće je odabrati jednu od tri opcije:
		
		\begin{packed_item}
			\item Korisnik teretana
			\item Admin teretana
			\item Trener
		\end{packed_item}
	
		
	
			
		\textit{Za pomoć pogledati reference navedene u poglavlju „Popis literature“, a po potrebi konzultirati sadržaj na internetu koji nudi dobre smjernice u tom pogledu.}
		\eject
		
		\section{Primjeri u \LaTeX u}
		
		\textit{Ovo potpoglavlje izbrisati.}\\

		U nastavku se nalaze različiti primjeri kako koristiti osnovne funkcionalnosti \LaTeX a koje su potrebne za izradu dokumentacije. Za dodatnu pomoć obratiti se asistentu na projektu ili potražiti upute na sljedećim web sjedištima:
		\begin{itemize}
			\item Upute za izradu diplomskog rada u \LaTeX u - \url{https://www.fer.unizg.hr/_download/repository/LaTeX-upute.pdf}
			\item \LaTeX\ projekt - \url{https://www.latex-project.org/help/}
			\item StackExchange za Tex - \url{https://tex.stackexchange.com/}\\
		
		\end{itemize} 	


		
		\noindent \underbar{podcrtani tekst}, \textbf{podebljani tekst}, 	\textit{nagnuti tekst}\\
		\noindent \normalsize primjer \large primjer \Large primjer \LARGE {primjer} \huge {primjer} \Huge primjer \normalsize
				
		\begin{packed_item}
			
			\item  primjer
			\item  primjer
			\item  primjer
			\item[] \begin{packed_enum}
				\item primjer
				\item[] \begin{packed_enum}
					\item[1.a] primjer
					\item[b] primjer
				\end{packed_enum}
				\item primjer
			\end{packed_enum}
			
		\end{packed_item}
		
		\noindent primjer url-a: \url{https://www.fer.unizg.hr/predmet/proinz/projekt}
		
		\noindent posebni znakovi: \# \$ \% \& \{ \} \_ 
		$|$ $<$ $>$ 
		\^{} 
		\~{} 
		$\backslash$ 
		
		\begin{longtabu} to \textwidth {|X[8, l]|X[8, l]|X[16, l]|} %definicija širine tablice, širine stupaca i poravnanje
			
			%definicija naslova tablice
			\hline \multicolumn{3}{|c|}{\textbf{naslov unutar tablice}}	 \\[3pt] \hline
			\endfirsthead
			
			%definicija naslova tablice prilikom prijeloma
			\hline \multicolumn{3}{|c|}{\textbf{naslov unutar tablice}}	 \\[3pt] \hline
			\endhead
			
			\hline 
			\endlastfoot
			
			\rowcolor{LightGreen}IDKorisnik & INT	&  	Lorem ipsum dolor sit amet, consectetur adipiscing elit, sed do eiusmod  	\\ \hline
			korisnickoIme	& VARCHAR &   	\\ \hline 
			email & VARCHAR &   \\ \hline 
			ime & VARCHAR	&  		\\ \hline 
			\cellcolor{LightBlue} primjer	& VARCHAR &   	\\ \hline 
			
		\end{longtabu}
		

		\begin{table}[H]
			
			\begin{longtabu} to \textwidth {|X[8, l]|X[8, l]|X[16, l]|} 
				
				\hline 
				\endfirsthead
				
				\hline 
				\endhead
				
				\hline 
				\endlastfoot
				
				\rowcolor{LightGreen}IDKorisnik & INT	&  	Lorem ipsum dolor sit amet, consectetur adipiscing elit, sed do eiusmod  	\\ \hline
				korisnickoIme	& VARCHAR &   	\\ \hline 
				email & VARCHAR &   \\ \hline 
				ime & VARCHAR	&  		\\ \hline 
				\cellcolor{LightBlue} primjer	& VARCHAR &   	\\ \hline 
				
				
			\end{longtabu}
	
			\caption{\label{tab:referencatablica} Naslov ispod tablice.}
		\end{table}
		
		
		%unos slike
		\begin{figure}[H]
			\includegraphics[scale=0.4]{slike/aktivnost.PNG} %veličina slike u odnosu na originalnu datoteku i pozicija slike
			\centering
			\caption{Primjer slike s potpisom}
			\label{fig:promjene}
		\end{figure}
		
		\begin{figure}[H]
			\includegraphics[width=.9\linewidth]{slike/aktivnost.PNG} %veličina u odnosu na širinu linije
			\caption{Primjer slike s potpisom 2}
			\label{fig:promjene2} %label mora biti drugaciji za svaku sliku
		\end{figure}
		
		Referenciranje slike \ref{fig:promjene2} u tekstu.
		
		\eject
		
	