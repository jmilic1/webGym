\chapter{Specifikacija programske potpore}
		
	\section{Funkcionalni zahtjevi}
			
			\textbf{\textit{dio 1. revizije}}\\
			
			\textit{Navesti \textbf{dionike} koji imaju \textbf{interes u ovom sustavu} ili  \textbf{su nositelji odgovornosti}. To su prije svega korisnici, ali i administratori sustava, naručitelji, razvojni tim.}\\
				
			\textit{Navesti \textbf{aktore} koji izravno \textbf{koriste} ili \textbf{komuniciraju sa sustavom}. Oni mogu imati inicijatorsku ulogu, tj. započinju određene procese u sustavu ili samo sudioničku ulogu, tj. obavljaju određeni posao. Za svakog aktora navesti funkcionalne zahtjeve koji se na njega odnose.}\\
			
			
			\noindent \textbf{Dionici:}
			
			\begin{packed_enum}
				
				\item Dionik 1
				\item Dionik 2				
				\item ...
				
			\end{packed_enum}
			
			\noindent \textbf{Aktori i njihovi funkcionalni zahtjevi:}
			
			
			\begin{packed_enum}
				\item  \underbar{Aktor 1 (inicijator) može:}
				
				\begin{packed_enum}
					
					\item funkcionalnost 1
					\item funkcionalnost 2
					\begin{packed_enum}
						
						\item  podfunkcionalnost 1 
						\item  podfunkcionalnost 2
				
					\end{packed_enum}
					\item  funkcionalnost 3
					
				\end{packed_enum}
			
				\item  \underbar{Aktor 2 (sudionik) može:}
				
				\begin{packed_enum}
					
					\item funkcionalnost 1
					\item funkcionalnost 2
					
				\end{packed_enum}
			\end{packed_enum}
			
			\eject 
			
			
				
			\subsection{Obrasci uporabe}
				
				\textbf{\textit{dio 1. revizije}}
				
				\subsubsection{Opis obrazaca uporabe}
					\textit{Funkcionalne zahtjeve razraditi u obliku obrazaca uporabe. Svaki obrazac je potrebno razraditi prema donjem predlošku. Ukoliko u nekom koraku može doći do odstupanja, potrebno je to odstupanje opisati i po mogućnosti ponuditi rješenje kojim bi se tijek obrasca vratio na osnovni tijek.}\\
					
					
				\noindent \underbar{\textbf{UC11 - Plaćanje narudžbe}}
					\begin{packed_item}
	
						\item \textbf{Glavni sudionik: } Klijent
						\item  \textbf{Cilj:} Platiti željenu narudžbu
						\item  \textbf{Sudionici:} Baza podataka
						\item  \textbf{Preduvjet:} Korisnik je prijavljen i napravio je narudžbu
						\item  \textbf{Opis osnovnog tijeka:}
						
						\item[] \begin{packed_enum}
	
							\item Prikazuju se podaci o narudžbi
							\item Klijent odabire opciju dovrši narudžbu
						\end{packed_enum}
						
						\item  \textbf{Opis mogućih odstupanja:}
						
						\item[] \begin{packed_item}
	
							\item[-] Nedovoljno stanje računa klijenta
							\item[] \begin{packed_enum}
								
								\item Sustav obavještava klijenta da nema dovoljno sredstava na računu
								
							\end{packed_enum}
							
						\end{packed_item}
					\end{packed_item}
				
				\noindent \underbar{\textbf{UC12 - Pregled informacija o određenom treneru}}
					\begin{packed_item}
	
						\item \textbf{Glavni sudionik: } Registrirani i neregistrirani korisnici
						\item  \textbf{Cilj:} Dobiti informacije odabranog trenera
						\item  \textbf{Sudionici:} Baza podataka
						\item  \textbf{Preduvjet:} -
						\item  \textbf{Opis osnovnog tijeka:}
						
						\item[] \begin{packed_enum}
	
							\item Odabrati određenog trenera
							\item Prikaz osnovnih informacija (ime, prezime, popis teretana u kojima radi)
							\item Prikaz ponuda treninga i plana prehrane
						\end{packed_enum}
						
					\end{packed_item}
				
				\noindent \underbar{\textbf{UC13 - Odabir trenerovog programa treniranja ili plana prehrane}}
					\begin{packed_item}
	
						\item \textbf{Glavni sudionik: } Klijent
						\item  \textbf{Cilj:} Odabrati trenerov programa treniranja ili plana prehrane
						\item  \textbf{Sudionici:} Baza podataka, Trener
						\item  \textbf{Preduvjet:} Korisnik je prijavljen
						\item  \textbf{Opis osnovnog tijeka:}
						
						\item[] \begin{packed_enum}
	
							\item Klijent odabire određenu ponudu treninga ili plana prehrane
							\item Ima opciju izrade narudžbe
							\item Klijent potvrđuje narzudžbu
						\end{packed_enum}
						
						\item  \textbf{Opis mogućih odstupanja:}
						
						\item[] \begin{packed_item}
	
							\item[-] Klijent pokušava rezervirati program treninga koji je popunjena
							\item[] \begin{packed_enum}
								
								\item Sustav javlja da je termin popunjen i nudi mu slobodne termine
								
							\end{packed_enum}
							
						\end{packed_item}
					\end{packed_item}
					
					\noindent \underbar{\textbf{UC14 - Pregled, uređivanje i postavljanje vlastitih ciljeva}}
					\begin{packed_item}
	
						\item \textbf{Glavni sudionik: } Klijent
						\item  \textbf{Cilj:} Pregled, uređivanje i postavljanje vlastitih ciljeva
						\item  \textbf{Sudionici:} Baza podataka
						\item  \textbf{Preduvjet:} Korisnik je prijavljen i odabrao je promjenu svojih podataka
						\item  \textbf{Opis osnovnog tijeka:}
						
						\item[] \begin{packed_enum}
	
							\item Odabire opciju “Moji ciljevi”
							\item Pregled postojećih ciljeva
							\item Postavljanje novih ciljeva i izmjena postojećih (može ga se izmijeniti ili postavit kao odrađenog)
						\end{packed_enum}
						
					\end{packed_item}
				
				\noindent \underbar{\textbf{UC15 - Uređivanje trenerove ponude za treninge}}
					\begin{packed_item}
	
						\item \textbf{Glavni sudionik: } Trener
						\item  \textbf{Cilj:} Uređivanje ponude za treninge
						\item  \textbf{Sudionici:} Baza podataka
						\item  \textbf{Preduvjet:} Trener je prijavljen i odabrao je promjenu svojih podataka
						\item  \textbf{Opis osnovnog tijeka:}
						
						\item[] \begin{packed_enum}
	
							\item Trener uređuje svoju ponudu
							\item Izmjena se pohrani
							\item Baza podataka se ažurira
						\end{packed_enum}
						
					\end{packed_item}
				
				\noindent \underbar{\textbf{UC16 - Uređivanje trenerove ponude za plan prehrane}}
					\begin{packed_item}
	
						\item \textbf{Glavni sudionik: } Trener
						\item  \textbf{Cilj:} Uređivanje ponude za plan prehrane
						\item  \textbf{Sudionici:} Baza podataka
						\item  \textbf{Preduvjet:} Trener je prijavljen i odabrao je promjenu svojih podataka
						\item  \textbf{Opis osnovnog tijeka:}
						
						\item[] \begin{packed_enum}
	
							\item Trener uređuje svoju ponudu
							\item Izmjena se pohrani
							\item Baza podataka se ažurira
						\end{packed_enum}
						
					\end{packed_item}
				
				\noindent \underbar{\textbf{UC17 - Dozvola za rad trenera u teretani}}
					\begin{packed_item}
	
						\item \textbf{Glavni sudionik: } Trener
						\item  \textbf{Cilj:} Dobiti dozvolu za rad u određenoj teretani
						\item  \textbf{Sudionici:} Baza podataka, Voditelji odabrane teretane
						\item  \textbf{Preduvjet:} Trener je prijavljen, odabrana željena teretana
						\item  \textbf{Opis osnovnog tijeka:}
						
						\item[] \begin{packed_enum}
						
	                        \item Trener odabire opciju za pisanje zamolbe
							\item Trener piše zamolbu
							\item Trener šalje zamolbu za dozvolu za rad u toj teretani prehrane
						\end{packed_enum}
						
						\item  \textbf{Opis mogućih odstupanja:}
						
						\item[] \begin{packed_item}
	
							\item[-] Trener je već zaposlen u toj teretani
							\item[] \begin{packed_enum}
								
								\item Sustav mu javlja da je već zaposlen u toj teretani
								
							\end{packed_enum}
							
						\end{packed_item}
					\end{packed_item}
				
				\noindent \underbar{\textbf{UC18 - Pregled postojećih klijenata}}
					\begin{packed_item}
	
						\item \textbf{Glavni sudionik: } Trener
						\item  \textbf{Cilj:} Pregled postojećih klijenata i popis zajedničkih termina treninga
						\item  \textbf{Sudionici:} Baza podataka
						\item  \textbf{Preduvjet:} Trener je prijavljen, odabran pregled podataka
						\item  \textbf{Opis osnovnog tijeka:}
						
						\item[] \begin{packed_enum}
						
	                        \item Trener odabire opciju za popis klijenata
							\item Treneru se prikazuje cijeli popis klijenata i za svakog popis zajedničkih termina treninga
						\end{packed_enum}
						
					\end{packed_item}
					
				\noindent \underbar{\textbf{UC19 - Pregled podataka određenog klijenta}}
					\begin{packed_item}
	
						\item \textbf{Glavni sudionik: } Trener
						\item  \textbf{Cilj:} Pogledati podatke određenog klijenta
						\item  \textbf{Sudionici:} Baza podataka
						\item  \textbf{Preduvjet:} Trener je prijavljen, odabran pregled podataka
						\item  \textbf{Opis osnovnog tijeka:}
						
						\item[] \begin{packed_enum}
						
	                        \item Trener odabrire određenog klijenta
							\item Treneru se prikazuju osobni podatci klijenta
						\end{packed_enum}
						
					\end{packed_item}
					
				\subsubsection{Dijagrami obrazaca uporabe}
					
					\textit{Prikazati odnos aktora i obrazaca uporabe odgovarajućim UML dijagramom. Nije nužno nacrtati sve na jednom dijagramu. Modelirati po razinama apstrakcije i skupovima srodnih funkcionalnosti.}
				\eject		
				
			\subsection{Sekvencijski dijagrami}
				
				\textbf{\textit{dio 1. revizije}}\\
				
				\textit{Nacrtati sekvencijske dijagrame koji modeliraju najvažnije dijelove sustava (max. 4 dijagrama). Ukoliko postoji nedoumica oko odabira, razjasniti s asistentom. Uz svaki dijagram napisati detaljni opis dijagrama.}
				\eject
	
		\section{Ostali zahtjevi}
		
			\textbf{\textit{dio 1. revizije}}\\
		 
			 \textit{Nefunkcionalni zahtjevi i zahtjevi domene primjene dopunjuju funkcionalne zahtjeve. Oni opisuju \textbf{kako se sustav treba ponašati} i koja \textbf{ograničenja} treba poštivati (performanse, korisničko iskustvo, pouzdanost, standardi kvalitete, sigurnost...). Primjeri takvih zahtjeva u Vašem projektu mogu biti: podržani jezici korisničkog sučelja, vrijeme odziva, najveći mogući podržani broj korisnika, podržane web/mobilne platforme, razina zaštite (protokoli komunikacije, kriptiranje...)... Svaki takav zahtjev potrebno je navesti u jednoj ili dvije rečenice.}
			 
			 
			 
	