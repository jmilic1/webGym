\chapter{Specifikacija programske potpore}
		
<<<<<<< HEAD
		\section{Funkcionalni zahtjevi}
		
		
		\noindent \textbf{Dionici:}
		
		\begin{packed_enum}
			
			\item Klijent teretane
			\begin{packed_enum}
				
				\item registrirani
				\item neregistrirani
				
			\end{packed_enum}
			\item Trener
			\item Voditelj teretane			
			\item Administrator
			\item Razvojni tim
			
		\end{packed_enum}
		
		\noindent \textbf{Aktori i njihovi funkcionalni zahtjevi:}
		
		
		\begin{packed_enum}
			\item  \underbar{Neregistrirani/neprijavljeni korisnik (inicijator) može:}
			
			\begin{packed_enum}
				
				\item pregledati popis svih teretana na platformi
				\item sortirati spomenuti popis prema sljedećim kriterijima: ime teretane, lokacija, trener
				\item otvoriti početnu stranicu svake teretane na kojoj se nalaze osnovne informacije (radno vrijeme, lokacija, cijena
				članarine…)
				\item izraditi administratorski, voditeljski, trenerski ili korisnički račun s namjerom treniranja u teretani za koje je potrebno navesti ime, prezime i email adresu, dok se može, ali ne mora dodati PayPal račun te je za izradu trenerskog korisničkog računa posebno potrebno navesti posebne podatke poput visine i težine
				
				
			\end{packed_enum}
			
			\item  \underbar{Klijent (inicijator) može:}
			
			\begin{packed_enum}
				
				\item pregledavati i sortirati popis registriranih teretana
				\item pregledavati i mijenjati osobne podatke
				\item izbrisati svoj korisnički račun
				\item plaćati članarine u teretanama putem interneta
				\item pregledavati sve izvršene transakcije u kojima su sudjelovali
				\item pregledavati popis teretana u kojima smiju vježbati, odnosno u kojima su platili članarinu
				\item kupovati planove prehrane i vježbanja od trenera
				\item ugovarati privatne ili grupne treninge
				\item voditi i pratiti napredak u vlastitom planu vježbanja
				
			\end{packed_enum}
			
			\item  \underbar{Trener (inicijator) može:}
			
			\begin{packed_enum}
				
				\item pregledavati i sortirati popis registriranih teretana
				\item pregledavati i mijenjati osobne podatke
				\item izbrisati svoj korisnički račun
				\item objavljivati ponude planova treninga i/ili vježbanja
				\item objavljivati i ugovarati termine privatnih i grupnih treninga u teretanama gdje imaju te ovlasti
				\item pregledavati sve izvršene transakcije u kojima su sudjelovali
				\item pregledavati popis teretana u kojima smiju djelovati, odnosno raditi (voditi treninge, planovi prehrane i sl.)
				\item nuditi usluge treniranja teretanama
				
			\end{packed_enum}
			
			\item  \underbar{Voditelj teretane (inicijator) može:}
			
			\begin{packed_enum}
				
				\item pregledavati i sortirati popis registriranih teretana
				\item pregledavati i mijenjati osobne podatke
				\item izbrisati svoj korisnički račun
				\item stvarati nove teretane u sustavu
				\item davati dozvolu drugim voditeljima da vode neke njegove teretane
				\item mijenjati važne informacije o teretanama (radno vrijeme, lokacija i sl.)
				\item dopuštati registriranim trenerima rad u teretanama koje vodi
				\item vidjeti sve izvršene transakcije na aplikaciji unutar vlastite teretane
				
			\end{packed_enum}
			
			\item  \underbar{Administrator (inicijator) može:}
			
			\begin{packed_enum}
				
				\item pregledavati i sortirati popis registriranih teretana
				\item pregledavati i mijenjati osobne podatke
				\item vidjeti sve korisničke račune
				\item izbrisati svoj korisnički račun
				\item stvarati nove i brisati postojeće teretane u sustavu
				\item pregledati sve izvršene transakcije u aplikaciji
				\item davati dozvolu  voditeljima da vode pojedine teretane
				\item mijenjati važne informacije o teretanama (radno vrijeme, lokacija i sl.)
				\item dopuštati registriranim trenerima rad u teretanama
				
			\end{packed_enum}
			
			
			\item  \underbar{Baza podataka (sudionik):}
			
			\begin{packed_enum}
				
				\item pohranjuje sve podatke o korisnicima
				\item čuva informacije o ulogama pojedinih korisnika
				\item pohranjuje podatke o svim teretanama, njihovim voditeljima, trenerima i članovima
				\item pohranjuje izvršene transakcije
				
			\end{packed_enum}
			
		\end{packed_enum}
		
		\eject

=======
	\section{Funkcionalni zahtjevi}
			
			
			\noindent \textbf{Dionici:}
			
			\begin{packed_enum}
				
				\item Klijent teretane
				\begin{packed_enum}
					
					\item registrirani
					\item neregistrirani
					
				\end{packed_enum}
				\item Trener
				\item Voditelj teretane			
				\item Administrator
				\item Razvojni tim
				
			\end{packed_enum}
			
			\noindent \textbf{Aktori i njihovi funkcionalni zahtjevi:}
			
			
			\begin{packed_enum}
				\item  \underbar{Neregistrirani/neprijavljeni korisnik (inicijator) može:}
				
				\begin{packed_enum}
					
					\item pregledati popis svih teretana na platformi
					\item sortirati spomenuti popis prema sljedećim kriterijima: ime teretane, lokacija, trener
					\item otvoriti početnu stranicu svake teretane na kojoj se nalaze osnovne informacije (radno vrijeme, lokacija, cijena
					članarine…)
					\item izraditi administratorski, voditeljski, trenerski ili korisnički račun s namjerom treniranja u teretani za koje je potrebno navesti ime, prezime i email adresu, dok se može, ali ne mora dodati PayPal račun te je za izradu trenerskog korisničkog računa posebno potrebno navesti posebne podatke poput visine i težine
					
					
				\end{packed_enum}
			
				\item  \underbar{Klijent (inicijator) može:}
				
				\begin{packed_enum}
					
					\item pregledavati i sortirati popis registriranih teretana
					\item pregledavati i mijenjati osobne podatke
					\item izbrisati svoj korisnički račun
					\item plaćati članarine u teretanama putem interneta
					\item pregledavati sve izvršene transakcije u kojima su sudjelovali
					\item pregledavati popis teretana u kojima smiju vježbati, odnosno u kojima su platili članarinu
					\item kupovati planove prehrane i vježbanja od trenera
					\item ugovarati privatne ili grupne treninge
					\item voditi i pratiti napredak u vlastitom planu vježbanja
					
				\end{packed_enum}
			
				\item  \underbar{Trener (inicijator) može:}
				
				\begin{packed_enum}
					
					\item pregledavati i sortirati popis registriranih teretana
					\item pregledavati i mijenjati osobne podatke
					\item izbrisati svoj korisnički račun
					\item objavljivati ponude planova treninga i/ili vježbanja
					\item objavljivati i ugovarati termine privatnih i grupnih treninga u teretanama gdje imaju te ovlasti
					\item pregledavati sve izvršene transakcije u kojima su sudjelovali
					\item pregledavati popis teretana u kojima smiju djelovati, odnosno raditi (voditi treninge, planovi prehrane i sl.)
					\item nuditi usluge treniranja teretanama
					
				\end{packed_enum}
		
				\item  \underbar{Voditelj teretane (inicijator) može:}
				
				\begin{packed_enum}
					
					\item pregledavati i sortirati popis registriranih teretana
					\item pregledavati i mijenjati osobne podatke
					\item izbrisati svoj korisnički račun
					\item stvarati nove teretane u sustavu
					\item davati dozvolu drugim voditeljima da vode neke njegove teretane
					\item mijenjati važne informacije o teretanama (radno vrijeme, lokacija i sl.)
					\item dopuštati registriranim trenerima rad u teretanama koje vodi
					\item vidjeti sve izvršene transakcije na aplikaciji unutar vlastite teretane
					
				\end{packed_enum}
	
				\item  \underbar{Administrator (inicijator) može:}
				
				\begin{packed_enum}
					
					\item pregledavati i sortirati popis registriranih teretana
					\item pregledavati i mijenjati osobne podatke
					\item vidjeti sve korisničke račune
					\item izbrisati svoj korisnički račun
					\item stvarati nove i brisati postojeće teretane u sustavu
					\item pregledati sve izvršene transakcije u aplikaciji
					\item davati dozvolu  voditeljima da vode pojedine teretane
					\item mijenjati važne informacije o teretanama (radno vrijeme, lokacija i sl.)
					\item dopuštati registriranim trenerima rad u teretanama
					
				\end{packed_enum}


				\item  \underbar{Baza podataka (sudionik):}
				
				\begin{packed_enum}
					
					\item pohranjuje sve podatke o korisnicima
					\item čuva informacije o ulogama pojedinih korisnika
					\item pohranjuje podatke o svim teretanama, njihovim voditeljima, trenerima i članovima
					\item pohranjuje izvršene transakcije
					
				\end{packed_enum}
		
			\end{packed_enum}
			
			\eject 
>>>>>>> devdoc
			
			
				
			\subsection{Obrasci uporabe}
				
				\textbf{\textit{dio 1. revizije}}
				
				\subsubsection{Opis obrazaca uporabe}
<<<<<<< HEAD
					

					\noindent \underbar{\textbf{UC1 - Pregled teretana}}
					\begin{packed_item}
	
						\item \textbf{Glavni sudionik: }neregistrirani korisnik, klijent, trener, voditelj teretane i administrator
						\item  \textbf{Cilj:} Pregled teretana
						\item  \textbf{Preduvjet:} -
						\item  \textbf{Opis osnovnog tijeka:}
						
						\item[] \begin{packed_enum}
	
							\item Prikazan popis teretana
							\item Korisnik može tražit teretane (prema nekim kriterijima - search)
							\item Korisnik može birati teretanu o kojoj će dobiti informacije
						\end{packed_enum}
						\end{packed_item}
						
						
						\noindent \underbar{\textbf{UC2 - Registracija korisnika}}
					\begin{packed_item}
	
						\item \textbf{Glavni sudionik: }Neregistrirani korisnik
						\item  \textbf{Cilj:} Izrada korisničkog računa kojim korisnik dobiva dodatne funkcionalnosti sustava
						\item  \textbf{Sudionici:} Baza podataka
						\item  \textbf{Preduvjet:} -
=======
					\textit{Funkcionalne zahtjeve razraditi u obliku obrazaca uporabe. Svaki obrazac je potrebno razraditi prema donjem predlošku. Ukoliko u nekom koraku može doći do odstupanja, potrebno je to odstupanje opisati i po mogućnosti ponuditi rješenje kojim bi se tijek obrasca vratio na osnovni tijek.}\\
					

					\noindent \underbar{\textbf{UC$<$broj obrasca$>$ -$<$ime obrasca$>$}}
					\begin{packed_item}
	
						\item \textbf{Glavni sudionik: }$<$sudionik$>$
						\item  \textbf{Cilj:} $<$cilj$>$
						\item  \textbf{Sudionici:} $<$sudionici$>$
						\item  \textbf{Preduvjet:} $<$preduvjet$>$
>>>>>>> devdoc
						\item  \textbf{Opis osnovnog tijeka:}
						
						\item[] \begin{packed_enum}
	
<<<<<<< HEAD
							\item Neprijavljeni korisnik odabire opciju registracije
							\item Neprijavljeni korisnik unosi potrebne podatke
							\item Korisnik prima obavijest o uspješnoj registraciji
						\end{packed_enum}
					
						\item  \textbf{Opis mogućih odstupanja:}
						
						\item[] \begin{packed_item}
	
							\item[-]
							Odabir zauzetog korisničkog imena ili e-maila, odabir nepostojećeg e-maila ili nedozvoljen format unosa nekog od podataka
							\item[] \begin{packed_enum}
								
								\item Sustav neprijavljenom korisniku šalje objavu o neuspješnoj registraciji te ga vraća na početnu stranicu za registraciju
								\item Korisnik mijenja neispravne podatke i završava unos ili odustaje od registriranja
								
							\end{packed_enum}
						\end{packed_item}
					\end{packed_item}
					
					\noindent \underbar{\textbf{UC3 - Prijava u sustav}}
					\begin{packed_item}
	
						\item \textbf{Glavni sudionik: }Klijent
						\item  \textbf{Cilj:} Dobiti pristup korisničkom sučelju
						\item  \textbf{Sudionici:} Baza podataka
						\item  \textbf{Preduvjet:} Registracija
						\item  \textbf{Opis osnovnog tijeka:}
						
						\item[] \begin{packed_enum}
	
							\item Unos korisničkog imena i lozinke
							\item Potvrda o ispravnosti unesenih podataka
							\item Pristup korisničkim funkcionalnostima
						\end{packed_enum}
					
=======
							\item $<$opis korak jedan$>$
							\item $<$opis korak dva$>$
							\item $<$opis korak tri$>$
							\item $<$opis korak četiri$>$
							\item $<$opis korak pet$>$
						\end{packed_enum}
						
>>>>>>> devdoc
						\item  \textbf{Opis mogućih odstupanja:}
						
						\item[] \begin{packed_item}
	
<<<<<<< HEAD
							\item[-]
							Neispravno korisničko ime i/ili lozinka
							\item[] \begin{packed_enum}
								
								\item Sustav obavještava korisnika o neuspješnom upisu i vraća ga na stranicu za prijavu
							\end{packed_enum}
						\end{packed_item}
					\end{packed_item}
					
					\noindent \underbar{\textbf{UC4 - Pregled osobnih podataka}}
					\begin{packed_item}
	
						\item \textbf{Glavni sudionik: }Klijent
						\item  \textbf{Cilj:} Pregledati osobne podatke korisnika
						\item  \textbf{Sudionici:} Baza podataka
						\item  \textbf{Preduvjet:} Klijent je prijavljen
						\item  \textbf{Opis osnovnog tijeka:}
						
						\item[] \begin{packed_enum}
	
							\item Klijent odabire opciju "Osobni podaci"
							\item Aplikacija prikazuje osobne podatke korisnika
						\end{packed_enum}
					\end{packed_item}
					
					
					\noindent \underbar{\textbf{UC5 - Promjena osobnih podataka}}
					\begin{packed_item}
	
						\item \textbf{Glavni sudionik: }Klijent, trener, voditelj, administrator
						\item  \textbf{Cilj:} Promijeniti osobne podatke
						\item  \textbf{Sudionici:} Baza podataka
						\item  \textbf{Preduvjet:} Korisnik je prijavljen
						\item  \textbf{Opis osnovnog tijeka:}
						
						\item[] \begin{packed_enum}
	
							\item Korisnik odabire opciju promjene osobnih podataka
							\item Korisnik mijenja svoje osobne podatke
							\begin{packed_item}
	
							\item Ako je korsnik prijavljen kao običan korisnik, on može postaviti svoje ciljeve i rezultate
							\item Ako je korisnik prijavljen kao trener, on može uređivati svoju stranicu
						\end{packed_item}
							\item Korisnik bira opciju "Spremi promjenu"
							\item Ažuracija baze podataka
						\end{packed_enum}
					
						\item  \textbf{Opis mogućih odstupanja:}
						
						\item[] \begin{packed_item}
	
							\item[-]
						Korisnik promijeni podatke, ali ne odabere opciju "Spremi promjenu"
							\item[] \begin{packed_enum}
								
								\item Sustav obavještava korisnika da nije spremio podatke prije izlaska iz prozora
							\end{packed_enum}
						\end{packed_item}
					\end{packed_item}
					
					\noindent \underbar{\textbf{UC6 - Brisanje korisničkog računa}}
					\begin{packed_item}
	
						\item \textbf{Glavni sudionik: }Klijent, trener, voditelj, administrator
						\item  \textbf{Cilj:} Izbrisati svoj korisnički račun
						\item  \textbf{Sudionici:} Baza podataka
						\item  \textbf{Preduvjet:} Korisnik je prijavljen
						\item  \textbf{Opis osnovnog tijeka:}
						
						\item[] \begin{packed_enum}
	
							\item Korisnik pregledava osobne podatke
							\item Korisnik bira opciju "Obriši račun"
							\item Korisnik briše račun
							\item Korisnikov račun se briše iz baze podataka
							\item Otvara se početna stranica
						\end{packed_enum}
					\end{packed_item}
					
					\noindent \underbar{\textbf{UC7 - Pregled specifične teretane}}
					\begin{packed_item}
	
						\item \textbf{Glavni sudionik: }Neregistrirani korisnik, klijent, trener, voditelj, administrator
						\item  \textbf{Cilj:} Vidjeti osnovne podatke o teretani i trenere u toj teretani
						\item  \textbf{Sudionici:} Baza podataka
						\item  \textbf{Preduvjet:} -
						\item  \textbf{Opis osnovnog tijeka:}
						
						\item[] \begin{packed_enum}
	
							\item Korisnik odabire željenu teretanu
							\item Prikazuju se voditelji teretane, treneri koji su dio te teretane, lokacija i radno vrijeme te ponuda članarine
						\end{packed_enum}
					\end{packed_item}
					
					\noindent \underbar{\textbf{UC8 - Pregled transakcija}}
					\begin{packed_item}
	
						\item \textbf{Glavni sudionik: }Klijent, trener, voditelj
						\item  \textbf{Cilj:} Pregled transakcija u kojima je korisnik do sada sudjelovao
						\item  \textbf{Sudionici:} Baza podataka
						\item  \textbf{Preduvjet:} Korisnik je prijavljen
						\item  \textbf{Opis osnovnog tijeka:}
						
						\item[] \begin{packed_enum}
	                        \item Korisnik odabire opciju pregleda osobnih podataka
							\item Korisnik odabire opciju pregleda transakcija
							\item Korisnik dobiva prikaz svih transakcija u kojima je sudjelovao
						\end{packed_enum}
					\end{packed_item}
					
					\noindent \underbar{\textbf{UC9 - Pregled određene transakcije}}
					\begin{packed_item}
	
						\item \textbf{Glavni sudionik: }Klijent, trener, voditelj
						\item  \textbf{Cilj:} Pregled sudionika u transakciji, opis usluge, iznos plaćanja u kunama i datum izvršenja transakcije
						\item  \textbf{Sudionici:} Baza podataka
						\item  \textbf{Preduvjet:} Korisnik je prijavljen
						\item  \textbf{Opis osnovnog tijeka:}
						
						\item[] \begin{packed_enum}
							\item Korisnik odabire opciju pregleda transakcija
							\item Korisnik odabire opciju pregleda određene transakcije
							\item Pregled detalja odabrane transakcije
						\end{packed_enum}
					\end{packed_item}
					
					\noindent \underbar{\textbf{UC10 - Učlanjivanje korisnika u određenu teretanu}}
					\begin{packed_item}
	
						\item \textbf{Glavni sudionik: }Klijent
						\item  \textbf{Cilj:} Učlanjivanje korisnika u određenu teretanu (ili lanac teretana)
						\item  \textbf{Sudionici:} Baza podataka
						\item  \textbf{Preduvjet:} Korisnik je prijavljen
						\item  \textbf{Opis osnovnog tijeka:}
						
						\item[] \begin{packed_enum}
	
							\item Korisnik odabire određenu teretanu
							\item Korisniku se prikazuje ponuda vrsta članarina
							\item Korisnik odabire opciju članarine koju želi platiti
						\end{packed_enum}
					
						\item  \textbf{Opis mogućih odstupanja:}
						
						\item[] \begin{packed_item}
	
						\item[-]
						Korisnik pokušava kupiti članarinu u teretani u kojo već ima aktivnu članarinu istog tipa
							\item[] \begin{packed_enum}
								
								\item Sustav obavještava korisnika da već postoji aktivna članarina tog tipa
								\item Vraća ga na stranicu s popisom članarina te teretane
							\end{packed_enum}
=======
							\item[2.a] $<$opis mogućeg scenarija odstupanja u koraku 2$>$
							\item[] \begin{packed_enum}
								
								\item $<$opis rješenja mogućeg scenarija korak 1$>$
								\item $<$opis rješenja mogućeg scenarija korak 2$>$
								
							\end{packed_enum}
							\item[2.b] $<$opis mogućeg scenarija odstupanja u koraku 2$>$
							\item[3.a] $<$opis mogućeg scenarija odstupanja  u koraku 3$>$
							
>>>>>>> devdoc
						\end{packed_item}
					\end{packed_item}
				
					
				\subsubsection{Dijagrami obrazaca uporabe}
					
					\textit{Prikazati odnos aktora i obrazaca uporabe odgovarajućim UML dijagramom. Nije nužno nacrtati sve na jednom dijagramu. Modelirati po razinama apstrakcije i skupovima srodnih funkcionalnosti.}
				\eject		
				
			\subsection{Sekvencijski dijagrami}
				
				\textbf{\textit{dio 1. revizije}}\\
				
				\textit{Nacrtati sekvencijske dijagrame koji modeliraju najvažnije dijelove sustava (max. 4 dijagrama). Ukoliko postoji nedoumica oko odabira, razjasniti s asistentom. Uz svaki dijagram napisati detaljni opis dijagrama.}
				\eject
	
		\section{Ostali zahtjevi}
		
			\textbf{\textit{dio 1. revizije}}\\
		 
			 \textit{Nefunkcionalni zahtjevi i zahtjevi domene primjene dopunjuju funkcionalne zahtjeve. Oni opisuju \textbf{kako se sustav treba ponašati} i koja \textbf{ograničenja} treba poštivati (performanse, korisničko iskustvo, pouzdanost, standardi kvalitete, sigurnost...). Primjeri takvih zahtjeva u Vašem projektu mogu biti: podržani jezici korisničkog sučelja, vrijeme odziva, najveći mogući podržani broj korisnika, podržane web/mobilne platforme, razina zaštite (protokoli komunikacije, kriptiranje...)... Svaki takav zahtjev potrebno je navesti u jednoj ili dvije rečenice.}
			 
			 
			 
	