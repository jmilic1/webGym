\chapter{Specifikacija programske potpore}
		
	\section{Funkcionalni zahtjevi}
			
			
			\noindent \textbf{Dionici:}
			
			\begin{packed_enum}
				
				\item Klijent teretane
				\begin{packed_enum}
					
					\item registrirani
					\item neregistrirani
					
				\end{packed_enum}
				\item Trener
				\item Voditelj teretane			
				\item Administrator
				\item Razvojni tim
				
			\end{packed_enum}
			
			\noindent \textbf{Aktori i njihovi funkcionalni zahtjevi:}
			
			
			\begin{packed_enum}
				\item  \underbar{Neregistrirani/neprijavljeni korisnik (inicijator) može:}
				
				\begin{packed_enum}
					
					\item pregledati popis svih teretana na platformi
					\item sortirati spomenuti popis prema sljedećim kriterijima: ime teretane, lokacija, trener
					\item otvoriti početnu stranicu svake teretane na kojoj se nalaze osnovne informacije (radno vrijeme, lokacija, cijena
					članarine…)
					\item izraditi administratorski, voditeljski, trenerski ili korisnički račun s namjerom treniranja u teretani
					
				\end{packed_enum}
			
				\item  \underbar{Klijent (inicijator) može:}
				
				\begin{packed_enum}
					
					\item pregledavati i sortirati popis registriranih teretana
					\item pregledavati i mijenjati osobne podatke
					\item izbrisati svoj korisnički račun
					\item plaćati članarine u teretanama putem interneta
					\item pregledavati sve izvršene transakcije u kojima su sudjelovali
					\item pregledavati popis teretana u kojima smiju vježbati, odnosno u kojima su platili članarinu
					\item kupovati planove prehrane i vježbanja od trenera
					\item ugovarati privatne ili grupne treninge
					\item voditi i pratiti napredak u vlastitom planu vježbanja
					
				\end{packed_enum}
			
				\item  \underbar{Trener (inicijator) može:}
				
				\begin{packed_enum}
					
					\item pregledavati i sortirati popis registriranih teretana
					\item pregledavati i mijenjati osobne podatke
					\item izbrisati svoj korisnički račun
					\item objavljivati ponude planova treninga i/ili vježbanja
					\item objavljivati i ugovarati termine privatnih i grupnih treninga u teretanama gdje imaju te ovlasti
					\item pregledavati sve izvršene transakcije u kojima su sudjelovali
					\item pregledavati popis teretana u kojima smiju djelovati, odnosno raditi (voditi treninge, planovi prehrane i sl.)
					
				\end{packed_enum}
		
				\item  \underbar{Voditelj teretane (inicijator) može:}
				
				\begin{packed_enum}
					
					\item pregledavati i sortirati popis registriranih teretana
					\item pregledavati i mijenjati osobne podatke
					\item izbrisati svoj korisnički račun
					\item stvarati nove teretane u sustavu
					\item davati dozvolu drugim voditeljima da vode neke njegove teretane
					\item mijenjati važne informacije o teretanama (radno vrijeme, lokacija i sl.)
					\item dopuštati registriranim trenerima rad u teretanama koje vodi
					\item vidjeti sve izvršene transakcije na aplikaciji unutar vlastite teretane
					
				\end{packed_enum}
	
				\item  \underbar{Administrator (inicijator) može:}
				
				\begin{packed_enum}
					
					\item pregledavati i sortirati popis registriranih teretana
					\item pregledavati i mijenjati osobne podatke
					\item vidjeti sve korisničke račune
					\item izbrisati svoj korisnički račun
					\item stvarati nove i brisati postojeće teretane u sustavu
					\item pregledati sve izvršene transakcije u aplikaciji
					\item davati dozvolu  voditeljima da vode pojedine teretane
					\item mijenjati važne informacije o teretanama (radno vrijeme, lokacija i sl.)
					\item dopuštati registriranim trenerima rad u teretanama
					
				\end{packed_enum}


				\item  \underbar{Baza podataka (sudionik):}
				
				\begin{packed_enum}
					
					\item pohranjuje sve podatke o korisnicima
					\item čuva informacije o ulogama pojedinih korisnika
					\item pohranjuje podatke o svim teretanama, njihovim voditeljima, trenerima i članovima
					
				\end{packed_enum}
		
			\end{packed_enum}
			
			\eject 
			
			
				
			\subsection{Obrasci uporabe}
				
				\textbf{\textit{dio 1. revizije}}
				
				\subsubsection{Opis obrazaca uporabe}
					\textit{Funkcionalne zahtjeve razraditi u obliku obrazaca uporabe. Svaki obrazac je potrebno razraditi prema donjem predlošku. Ukoliko u nekom koraku može doći do odstupanja, potrebno je to odstupanje opisati i po mogućnosti ponuditi rješenje kojim bi se tijek obrasca vratio na osnovni tijek.}\\
					

					\noindent \underbar{\textbf{UC$<$broj obrasca$>$ -$<$ime obrasca$>$}}
					\begin{packed_item}
	
						\item \textbf{Glavni sudionik: }$<$sudionik$>$
						\item  \textbf{Cilj:} $<$cilj$>$
						\item  \textbf{Sudionici:} $<$sudionici$>$
						\item  \textbf{Preduvjet:} $<$preduvjet$>$
						\item  \textbf{Opis osnovnog tijeka:}
						
						\item[] \begin{packed_enum}
	
							\item $<$opis korak jedan$>$
							\item $<$opis korak dva$>$
							\item $<$opis korak tri$>$
							\item $<$opis korak četiri$>$
							\item $<$opis korak pet$>$
						\end{packed_enum}
						
						\item  \textbf{Opis mogućih odstupanja:}
						
						\item[] \begin{packed_item}
	
							\item[2.a] $<$opis mogućeg scenarija odstupanja u koraku 2$>$
							\item[] \begin{packed_enum}
								
								\item $<$opis rješenja mogućeg scenarija korak 1$>$
								\item $<$opis rješenja mogućeg scenarija korak 2$>$
								
							\end{packed_enum}
							\item[2.b] $<$opis mogućeg scenarija odstupanja u koraku 2$>$
							\item[3.a] $<$opis mogućeg scenarija odstupanja  u koraku 3$>$
							
						\end{packed_item}
					\end{packed_item}
				
					
				\subsubsection{Dijagrami obrazaca uporabe}
					
					\textit{Prikazati odnos aktora i obrazaca uporabe odgovarajućim UML dijagramom. Nije nužno nacrtati sve na jednom dijagramu. Modelirati po razinama apstrakcije i skupovima srodnih funkcionalnosti.}
				\eject		
				
			\subsection{Sekvencijski dijagrami}
				
				\textbf{\textit{dio 1. revizije}}\\
				
				\textit{Nacrtati sekvencijske dijagrame koji modeliraju najvažnije dijelove sustava (max. 4 dijagrama). Ukoliko postoji nedoumica oko odabira, razjasniti s asistentom. Uz svaki dijagram napisati detaljni opis dijagrama.}
				\eject
	
		\section{Ostali zahtjevi}
		
			\textbf{\textit{dio 1. revizije}}\\
		 
			 \textit{Nefunkcionalni zahtjevi i zahtjevi domene primjene dopunjuju funkcionalne zahtjeve. Oni opisuju \textbf{kako se sustav treba ponašati} i koja \textbf{ograničenja} treba poštivati (performanse, korisničko iskustvo, pouzdanost, standardi kvalitete, sigurnost...). Primjeri takvih zahtjeva u Vašem projektu mogu biti: podržani jezici korisničkog sučelja, vrijeme odziva, najveći mogući podržani broj korisnika, podržane web/mobilne platforme, razina zaštite (protokoli komunikacije, kriptiranje...)... Svaki takav zahtjev potrebno je navesti u jednoj ili dvije rečenice.}
			 
			 
			 
	